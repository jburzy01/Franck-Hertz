\documentclass[aps, reprint,amsmath,amssymb]{revtex4-1} %APS Journal
\usepackage[T1]{fontenc}
\usepackage[utf8]{inputenc}
\usepackage{lmodern}
\usepackage{microtype}
\usepackage{graphicx}
\usepackage{siunitx}
\usepackage{bm}

\renewcommand{\vec}[1]{\boldsymbol{#1}}
\newcommand{\mat}[1]{\mathbf{#1}}
\newcommand{\uv}[1]{\vec{\hat{#1}}}
\newcommand{\x}{\vec{\hat{x}}}
\newcommand{\y}{\vec{\hat{y}}}
\newcommand{\z}{\vec{\hat{z}}}
\renewcommand{\d}{\partial}
\renewcommand{\L}{\mathcal{L}}
\renewcommand{\inf}{\infty}

\begin{document}
%----------------------------------------------------------------------
% title
%----------------------------------------------------------------------
\title{PHY64 Experiment 3: The Frank-Hertz Experiment}
\author{Matthew S. E. Peterson}
\author{Jackson Burzynski}
\affiliation{Department of Physics and Astronomy, Tufts University}
%\date{\today} 
\maketitle

%----------------------------------------------------------------------
% Body
%----------------------------------------------------------------------
\section{Introduction}
In 1914 German physicists James Frank and Gustav Hertz performed the first measurement to clearly show the quantum nature of atoms. By bombarding gases with energetic electrons, Frank and Hertz observed that the transfer of kinetic energy from the electrons could raise a mercury atom from its ground state to an excited state. The pair discovered that when an electron collided with a mercury atom, it could lose only a specific amount of its kinetic energy. Frank and Hertz were able to show that atoms absorb kinetic energy from electrons in quantized amounts. In a later experiment, they showed that the exited atoms radiate away the absorbed energy in these same amounts.

We replicate the Frank-Hertz experiment with an apparatus similar to the one used by Frank and Hertz in 1914. The central element of our experiment is a tube manufactured by Klinger consisting of three electrodes. The tube is evacuated except for the presence of a small amount of mercury. The cathode is heated by a filament to emit electrons. The electrons are then accelerated to the anode by a potential difference $V_a$. The anode is a mesh to allow for energetic electrons to pass through and reach the repeller. The current $I$ formed by the electrons reaching the repeller is measured by a picoameter. The collisions between electrons and mercury atoms result in a periodic variation of the repeller current. From the periodicity of this current we may infer the energy of the first excited state of the mercury atom.

\section{Theory}

When the accelerating voltage $V_a$ is small, the collisions between the electrons and the mercury atoms are completely elastic. As $V_a$ is increased, a threshold voltage is reached in which inelastic collisions between the electrons and the mercury atoms begin to occur. In these collisions, the electrons transfer almost of of their kinetic energy to the atom, causing a transition from the ground state to the first excited state. The electrons that have lost their kinetic energy in these collisions are then unable to reach the repeller, resulting in a decrease in the current $I$. As we continue to increase $V_a$, the scattered electrons are reaccelerated and are capable of overcoming the retarding current and reach the repeller resulting in an increase in the current. The current continues to increase until $V_a$ is large enough for an electron to have two inelastic collisions. As in the single-scattering scenario, these electrons have insufficient energy to reach the repeller, resulting in second decrease in the current $I$. This phenomenon continues as $V_a$ is increased to allow for higher collision multiplicities.

\section{Results}


\section{Error}

\section{Conclusion}


\end{document}
